% Einstellung von Variablen für das gesamte Dokument
\newcommand{\Ersteller}{__INPUT__}
\newcommand{\Kurs}{__INPUT__}
\newcommand{\Titel}{__INPUT__}
\newcommand{\Arbeitstyp}{__INPUT__} % z.B. Bachelorarbeit

% Formulierung der Tabelle auf dem Titelblatt
\newcommand{\Titelblatt}{
	\textbf{Verfasser:} & \Ersteller\\
	\textbf{Matrikelnummer:} & __INPUT__\\
	\textbf{Firma:} & __INPUT__\\
	\textbf{Abteilung:} & __INPUT__\\
	\textbf{Kurs:} & \Kurs\\
	\textbf{Studiengangsleiter:} & __INPUT__\\
	\textbf{Wissenschaftlicher Betreuer:} & __INPUT__ \flq{}\href{mailto:__INPUT[E-Mail]__}{__INPUT[E-Mail]__}\frq{}\\
	\textbf{Firmenbetreuer:} & __INPUT__ \flq{}\href{mailto:__INPUT[E-Mail]__}{__INPUT[E-Mail]__}\frq{}\\
	\textbf{Bearbeitungszeitraum:} & __INPUT[Datum]__ bis __INPUT[Datum]__\\
}

% Einfügen eines Sperrvermerkes:
\sperrfalse % oder \sperrtrue

% Anpassung der Pfade zu Bildern und Code
%\renewcommand{\Imgpath}{../img/}
%\renewcommand{\Codepath}{../code/}

% Mit hyphenation können die Regeln für die Silbentrennung überschrieben werden
% Der Input ist dazu eine mit Leerzeichen getrennte Liste
% (Das funktioniert aber nicht für Wörter mit Umlauten)
%\hyphenation{__INPUT__}