\documentclass[fontsize=12pt,listof=totoc]{scrreprt}

\newcommand{\Ersteller}{\textcolor{red}{<Ersteller>}}
\newcommand{\Kurs}{\textcolor{red}{<Kurs>}}
\newcommand{\Studiengang}{\textcolor{red}{<Studiengang>}}
\newcommand{\Profilfach}{\textcolor{red}{<Profilfach>}}
\newcommand{\Titel}{\textcolor{red}{<Der Titel dieser Arbeit>}}
\newcommand{\Arbeitstyp}{\textcolor{red}{<Arbeitstyp>}}
\newcommand{\Matrikelnummer}{\textcolor{red}{<1234567>}}
\newcommand{\Firma}{\textcolor{red}{<Firma>}}
\newcommand{\FirmenAdresse}{\textcolor{red}{<Stra{\ss}e> \\ <PLZ Ort>}}
\newcommand{\Abteilung}{\textcolor{red}{<Abteilung>}}
\newcommand{\Studiengangsleiter}{\textcolor{red}{<Studiengangsleiter>}}
\newcommand{\DHBetreuer}{\textcolor{red}{<Betreuer>}}
\newcommand{\DHBetreuerEmail}{\textcolor{red}{Betreuer@example.com}}
\newcommand{\Betreuer}{\textcolor{red}{<Firmenbetreuer>}}
\newcommand{\BetreuerEmail}{\textcolor{red}{Firmenbetreuer@example.com}}
\newcommand{\Zeitraum}{\textcolor{red}{<Anfang> bis <Ende>}}

\newcommand{\Abstractpath}{../text/abstract.tex}
\newcommand{\Textpath}{../text/text.tex}
\newcommand{\Imgpath}{../img/}
\newcommand{\Codepath}{../code/}
\newcommand{\Glossarypath}{../text/glossary.tex}
\newcommand{\Bibpath}{../bib/Quellen.bib}

\newif\ifsperr
\sperrfalse
\newif\ifglossary
\glossaryfalse
\newif\ifroman
\romanfalse

\newcommand{\Bibstyle}{style=authortitle}

\InputIfFileExists{../settings.tex}{}

\usepackage[T1]{fontenc}
\usepackage[utf8]{inputenc}
\usepackage[ngerman]{babel}
\usepackage{lmodern}
\usepackage{eurosym}
\usepackage{graphicx}
\usepackage{xcolor}
	\definecolor{gray}{rgb}{0.5,0.5,0.5}
	\definecolor{stringblue}{HTML}{2A00FF}
	\definecolor{taggreen}{HTML}{008080}
	\definecolor{attributepink}{HTML}{7F007F}
	\definecolor{jskeypink}{HTML}{7F0055}
	\definecolor{commentblue}{HTML}{3F5FBF}
	\definecolor{numberblue}{HTML}{33A1DE}
\usepackage[hyphens]{url}
\usepackage[strict]{csquotes}
\usepackage[backend=biber,\Bibstyle]{biblatex}
	\IfFileExists{\Bibpath}{
		\addbibresource{\Bibpath}
	}{}
\usepackage{setspace}
	\linespread{1.5}
\usepackage{accsupp}
	\newcommand\emptyaccsupp[1]{\BeginAccSupp{ActualText={}}#1\EndAccSupp{}}
	\DeclareRobustCommand\squelch[1]{%
		\BeginAccSupp{method=plain,ActualText={}}#1\EndAccSupp{}}
\usepackage{listings}
	\renewcommand{\ttdefault}{pcr}
	\lstset{
		basicstyle=\ttfamily, %\scriptsize
		stringstyle=\color{stringblue},
		keywordstyle=[1]\color{taggreen},
		keywordstyle=[2]\color{attributepink},
		keywordstyle=[3]\color{attributepink},
		keywordstyle=[4]\bfseries\color{jskeypink},
		keywordstyle=[5]\color{black},
		frame=single,
		breaklines=true,
		postbreak=\raisebox{0ex}[0ex][0ex]{\ensuremath{\color{red}\hookrightarrow\space}},
		numbers=left,
		numberstyle=\scriptsize\color{gray}\emptyaccsupp,
		stepnumber=1,
		tabsize=4,
		lineskip=1pt,
		captionpos=b,
		showstringspaces=false,
		literate=%
			*{0}{{{\color{numberblue}0}}}1
			{1}{{{\color{numberblue}1}}}1
			{2}{{{\color{numberblue}2}}}1
			{3}{{{\color{numberblue}3}}}1
			{4}{{{\color{numberblue}4}}}1
			{5}{{{\color{numberblue}5}}}1
			{6}{{{\color{numberblue}6}}}1
			{7}{{{\color{numberblue}7}}}1
			{8}{{{\color{numberblue}8}}}1
			{9}{{{\color{numberblue}9}}}1
	}
	\renewcommand\lstlistlistingname{Listingverzeichnis}
\usepackage[figure,table,lstlisting]{totalcount}
\usepackage{hyperref}
	\hypersetup{
		hidelinks,
		bookmarksopen=true,
		pdftitle=\Titel{},
		pdfauthor=\Ersteller{}
	}
	\def\sectionautorefname{Abschnitt}
	\def\chapterautorefname{Kapitel}
	\providecommand*{\lstnumberautorefname}{Listing}
	\pdfstringdef\plainDHBetreuerEmail{\DHBetreuerEmail}
\pdfstringdef\plainBetreuerEmail{\BetreuerEmail}
\usepackage{bookmark}
\usepackage{chngcntr}
	\counterwithout{footnote}{chapter} %Durchgehende Nummerierung der Fußnoten
\usepackage[headsepline,footsepline,plainfootsepline,automark]{scrpage2}
	\clearscrheadfoot
	\ihead[]{\squelch{\headmark}}
	\ifoot[\squelch{DHBW Mannheim}]{\squelch{DHBW Mannheim}}
	\cfoot[\squelch{- \pagemark{} -}]{\squelch{- \pagemark{} -}}
	\ofoot[\squelch{\Ersteller}]{\squelch{\Ersteller}}
	\setkomafont{pagehead}{\normalfont}
	\setkomafont{pagefoot}{\normalfont}
	\pagestyle{scrheadings}
\usepackage{scrhack}
\ifglossary
	\usepackage[toc,nonumberlist,translate=babel,acronym]{glossaries}
		\InputIfFileExists{\Glossarypath}{}
		\makeglossaries
\fi

\InputIfFileExists{../advanced_settings.tex}{}


\begin{document}
	\ifroman
		\pagenumbering{Roman}
	\fi
	\hypertarget{top}{}
	\bookmark[dest=top]{\Titel}
	\begin{titlepage}
		\includegraphics[height=70px]{img/firma}
		\hfill
		\includegraphics[height=70px]{img/dhbw}
		\begin{center}
			\large{Duale Hochschule Baden-Württemberg\\
			Mannheim}\\
			\usekomafont{subject}\Arbeitstyp\\
			\vspace{20px}
			\begin{onehalfspace}
				{\LARGE\sffamily\Titel\\}
			\end{onehalfspace}
			\vspace{30px}
			\normalsize
			\textbf{Studiengang \Studiengang}\\
			\textbf{\small{Profil \Profilfach}}\\
			\ifsperr
				\textcolor{red}{\textbf{-Sperrvermerk-}}\\
			\fi
			\vfill
		\end{center}
		\begin{tabular}{ll}
			\textbf{Verfasser:} & \Ersteller\\
			\textbf{Matrikelnummer:} & \Matrikelnummer\\
			\textbf{Firma:} & \Firma\\
			\textbf{Abteilung:} & \Abteilung\\
			\textbf{Kurs:} & \Kurs\\
			\textbf{Studiengangsleiter:} & \Studiengangsleiter \\
			\textbf{Wissenschaftlicher Betreuer:} & \DHBetreuer{} \flq{}\href{mailto:\plainDHBetreuerEmail}{\DHBetreuerEmail}\frq{}\\
			\textbf{Firmenbetreuer:} & \Betreuer{} \flq{}\href{mailto:\plainBetreuerEmail}{\BetreuerEmail}\frq{}\\
			\textbf{Bearbeitungszeitraum:} & \Zeitraum\\
		\end{tabular}
	\end{titlepage}
	
	\ifsperr
		\newpage
		\hypertarget{nda}{}
		\bookmark[level=chapter, dest=nda]{Sperrvermerk}
		\thispagestyle{scrplain}
		\chapter*{Sperrvermerk}
			Die nachfolgende Arbeit enthält vertrauliche Daten, welche geistiges Eigentum der\\[\baselineskip]
			\Firma \\
			\FirmenAdresse \\
			\linebreak
			darstellen.\\[\baselineskip]
			Sie darf als Leistungsnachweis des Studiengangs \enquote{\Studiengang} an der Dualen Hochschule Baden-Württemberg in Mannheim nur zu Prüfungszwecken zugänglich gemacht werden. Über den Inhalt ist Stillschweigen zu bewahren. Veröffentlichung oder Vervielfältigung der \Arbeitstyp{} – auch auszugsweise – ist ohne ausdrückliche Genehmigung der \Firma{} nicht gestattet. Die Wiedergabe von Gebrauchsnamen, Handelsnamen, Warenbezeichnungen usw. in dieser Arbeit berechtigt auch ohne besondere Kennzeichnung nicht zu der Annahme, dass solche Namen im Sinne der Warenzeichen- und Markenschutz-Gesetzgebung als frei zu betrachten wären und daher benutzt werden dürfen.\\
			\vfill
			\noindent
			\rule[-7px]{120px}{.4pt}\hfill\rule[-7px]{120px}{.4pt}\\
			Ort, Datum\hfill\makebox[120px][l]{Unterschrift}
	\fi
	
	\newpage
	\hypertarget{abstract}{}
	\bookmark[level=chapter, dest=abstract]{Kurzfassung}
	\thispagestyle{scrplain}
	\chapter*{Kurzfassung}
		\begin{tabular}{lp{0.8\columnwidth}}
			Verfasser: & \Ersteller \\
			Kurs: & \Kurs \\
			Thema: & \Titel \\
		\end{tabular}
		\\[\baselineskip]
		\InputIfFileExists{\Abstractpath}{}

	\newpage
		\hypertarget{toc}{}
		\bookmark[level=chapter, dest=toc]{\contentsname}
	\tableofcontents
		
	\iftotalfigures
		\listoffigures
	\fi
	
	\iftotallstlistings
		\lstlistoflistings
	\fi
		
	\iftotaltables
		\listoftables
	\fi

	\ifroman
		\newpage
		\newcounter{lastroman}
		\setcounter{lastroman}{\value{page}}
		\pagenumbering{arabic}
	\fi
	\InputIfFileExists{\Textpath}{}
	
	\ifroman
		\newpage
		\pagenumbering{Roman}
		\setcounter{page}{\value{lastroman}}
	\fi
	\ifglossary
		\printglossaries
	\fi

	\printbibliography[heading=bibintoc]{}
		
	\newpage
	\hypertarget{decl}{}
	\bookmark[level=chapter, dest=decl]{Ehrenwörtliche Erklärung}
	\chapter*{Ehrenwörtliche Erklärung}
		\thispagestyle{empty}
		Ich erkläre hiermit ehrenwörtlich:
		\renewcommand\labelenumi{\arabic{enumi})}
		\begin{enumerate}
			\item dass ich meine \Arbeitstyp{} mit dem Thema \\
				\textbf{\Titel} \\
				ohne fremde Hilfe angefertigt habe; 
			\item dass ich die Übernahme wörtlicher Zitate aus der Literatur sowie die Verwendung der Gedanken anderer Autoren an den entsprechenden Stellen innerhalb der \Arbeitstyp{} gekennzeichnet habe;
			\item dass ich meine \Arbeitstyp{} bei keiner anderen Prüfung vorgelegt habe;
			\item dass die eingereichte elektronische Fassung exakt mit der eingereichten schriftlichen Fassung übereinstimmt.\\
		\end{enumerate} 
		Ich bin mir bewusst, dass eine falsche Erklärung rechtliche Folgen haben wird.
		\vfill
		\noindent
		\rule[-7px]{120px}{.4pt}\hfill\rule[-7px]{120px}{.4pt}\\
		Ort, Datum\hfill\makebox[120px][l]{Unterschrift}
\end{document}